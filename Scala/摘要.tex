%---------------------摘要------------------------
%\noindent 取消首行缩进
% 中文摘要、关键字、分类号
\noindent \heiti 摘要:\songti 本文主要针对\LaTeX 为中文论文排版进行简要的描述,以工业工程该期刊的要求作为模板。 \\
\heiti 关键词:\songti \LaTeX ; 中文期刊 ;工业工程 \\
\heiti 中图分类号:\songti 000 \hspace{3cm}     \heiti 文献标识码:\songti A \\

% 英文标题居中加粗、作者、简介
\begin{center}
	\textbf{\zihao{4} How to Write An Chinese Journal with \LaTeX} \\ 
	\zihao{5} Joefsong \\ 
	\zihao{-5} (Heiheihei College Papapa School, QQ Group 970479548) \\
\end{center}

% 英文摘要和关键字
\noindent \textbf{Abstract:} Wo de ying yu hen lan, jiu bu fan yi le. Fan zheng wo jue de pai ban he fan yi ye mei you shen me guan xi. Dan shi wei le rang ying wen zhai yao nei rong xian de feng man hai shi yao duo xie yi dian. hao le wo bian bu xia qu le. Jiu zhe yang ba.\\
\textbf{key words:}  \LaTeX ; Chinese Journal; Industrial Engineering Journal \\

% 引言
现存的用来编辑或排版的工具很多,最被熟知的应该是微软的软件,笔者了解到国内商用(编辑部)比较多的应该是方正书版。但是微软的软件实现很多功能都需要插件,例如:word中插入公式所用到的内置公式编辑器较为繁琐,而Mathtype插件的兼容性很差,且需要收费才能使用正版插件,此外在处理文本的过程中经常出现格式问题,如无法与文本对齐等,而且当部分公式的字体(如表格内公式)需要调整字号的时候需要进入公式编辑器逐个调整;除此之外,在插入参考文献的时候需要借助Endnote 等工具;在作图的时候需要借助Viso等。而方正书版最大的缺陷就是收费,且价格不菲。因此完全免费且容易上手的排版软件\LaTeX 脱颖而出,并且英文期刊几乎全部都要求用\LaTeX 排版,部分中文期刊也都接受\LaTeX 排版的文章,少部分中文期刊要求用\LaTeX 排版,也有一部分中文期刊仅接受word的排版。因为工业工程期刊是接受\LaTeX 排版的论文的,并且提供了\LaTeX 的模板,但是由于版本问题,无法兼容现在的texlive。因此本文选择了工业工程这个期刊为例,按照其提供的论文排版要求,对该篇论文进行人工的排版。最终为需要利用\LaTeX 排版中文期刊论文的人士提供有效易学的方法。