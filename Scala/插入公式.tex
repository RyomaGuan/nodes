\section{插入公式}
	\LaTeX 相比于word 的优点之一便是公式的兼容性,本文仅对数学符号以及相关公式进行简要的介绍。
	\subsection{数学符号}
	在使用数学符号之前需要在导言区使用amssymb宏包,相关命令不予赘述,可以在texstudio 的符号中直接进行选择插入,无需记忆。需要注意的是,在使用数学符号的时候需要在 \$ \$ 内输入。 且由于反斜杠和 dollar 符号以及大括号等特殊符号与 \LaTeX 命令符号存在冲突,这些符号是 \LaTeX 的保留字符,因此在需要输入这些符号的情况下,有特殊的方式,可以参考下面三个实力,更多的保留字符可以自行搜索:\\
%	\hspace*{3cm}
	dollar 符号 ~~ \$  \hspace{5cm}	反斜杠 ~~ $\backslash$ \hspace{5cm}大括号    $\left\lbrace \right\rbrace$	 
	
	需要注意的是\$ \$ 内的符号都是默认的斜体,因此如需使用正体符号,需要重新定义。
	\subsection{常见公式} 
	常见的公式种类分为行内公式和行间公式,行内公式可以直接使用 \$ \$符号进行书写;而行间公式如$$a+b=c$$可通过\$\$ \quad \$\$符号实现,如果需要对公式进行标号以及交叉引用,可以通过equation环境实现。如果需要连续编辑多行公式则可以借助gather环境来实现。示例如下:
多行公式:
	\begin{gather}
		a+b=c \\
		\alpha + \beta =\gamma \notag \\
		x+y=z \notag
	\end{gather}
	方程组或者分段函数可以用cases环境实现:
	\begin{equation}
		\begin{cases}
			a+b=c \\
			\alpha + \beta =\gamma  \\
			x+y=z
		\end{cases} 
	\end{equation}
	\begin{equation}
		f(x)= 
		\begin{cases}
			ax+b+c+1 &if x<\gamma \\
			\alpha x + \beta &if x>\gamma 
		\end{cases} 
	\end{equation}
	长公式用split环境实现
	\begin{equation}
		\begin{split}
			f(x)&=10000000a^2x+123456789bx+1010101010c+1010101010+\alpha x + \beta + \gamma\\
			&=15746987\sqrt{\alpha x + \beta}+121212121212\lambda^{ax}x+\frac{1}{\lambda^2}\\
			&=0
		\end{split}
	\end{equation}
	
	由于矩阵的用法并不复杂,因此不再单独介绍,如需要了解,附录中有网友对矩阵输入整理的链接。