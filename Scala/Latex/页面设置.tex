%---------------------正文------------------------
%%%%%%%%%%%%%%%%%%%%%%%%%%%%%%%%%%%%%%%%%%%%%%%%%%%%%%%%%%%%%%%%%%%
\section{基本文本格式以及段落调整}	% 一级标题
期刊论文的要求中,要求最多的就是字体字号以及章节格式,因此本节简要介绍\LaTeX 中的中英文字体字号设置以及段落格式的基本要求的实现。

\subsection{字体和字号}	% 二级标题
\LaTeX 中若使用了ctex或者ctexart宏包,则需要注意\LaTeX 包含常用的四种中文字体以及三种英文字体可供选择。由于期刊论文一般不要求其他字体,因此本节不在讲述载入系统字体的方法。

\subsubsection{字体}%三级标题
\LaTeX 默认的中文字体为宋体, 且有黑体、 楷书和仿宋三种气体可供选择; 默认的英文字体为罗马字体 (Roman Family) ,且有无衬线字体 (Sans Serif Family) 以及打印字体 (Typewtiter Family) 可供选择。具体字体及其命令如下:

\begin{table}[htbp] \zihao{-5} 

	\caption{字体命令} \label{字体命令}
	\centering
	\begin{tabular}{l p{3cm}<{\raggedleft} l p{3cm}<{\centering}}
		\toprule
		中文字体&命令& 英文字体& 命令\\
		\hline
		宋体&	songti&	罗马字体& rmfamily\\
		黑体	& heiti &	无衬线字体 &	sffamily\\
		楷体 &	kaishu &	打印字体& 	ttfamily\\
		仿宋& fangsong & \\
		\hline
	\end{tabular}
\end{table}
	
\subsubsection{字号}
由于使用了中文的宏包,因此在设置字号的时候可以直接使用 zihao 命令进行定义字号大小。大括号中是字号的大小,如五号字体为5,小五号字体为-5。


\subsection{段落调整}
\subsubsection{基本段落格式}
需要注意的是\LaTeX 在编辑中文文本的时候会忽略空格和回车,两个中文字符之间不管有多少空格,生成的PDF文件都不会显示出来。此外单个回车也不会是生成全新段落,若要实现生成新的段落需要在段落间空一行。若仅需要换行而不需要生成新的段落可以用双反斜杠实现。利用空行生成的新段落会默认首行缩进两个字符,利用双反斜杠生成的新段落不会首行缩进。若想要取消某段落首行缩进,须在段首使用 noident 命令。而中文文本和英文文本在一起出现的 时候,会自动在两者之间生成空格。quad 一个空格  qquad 两个空格  

\subsubsection{行间距调整}
\LaTeX 使用中文ctex 宏包之后,默认的是1.3倍行距,而中文期刊不会对行距有特别的要求,因此可直接使用默认行距,若需要设置论文行距,可以在导言区重新定义baselinestretch的值,具体参见导言区命令。各级标题与段落之间的间距设置将在\ref{多级标题}%讲多级标题的章节
给出。

也会遇到情况如摘要和作者的间距太长,仅需提高摘要的位置,使其靠近作者或者标题,此实仅需利用vspace命令定义向上或向下移动的距离。常见距离单位如下:也可通过baselineskip 或 textwidth 等定义相对距离。
单位 	名称 	说明
mm 	毫米 	1 mm = 2.845 pt
pt 	点 	1 pt = 0.351 mm
cm 	厘米 	1 cm= 10 mm= 28.453 pt
in 	英寸 	1 in = 25.4 mm = 72.27 pt
ex 	ex 	1 ex = 当前字体尺寸中 x 的高度
em 	em 	1 em = 当前字体尺寸中 M 的宽度

\subsubsection{自动编号} \label{自动编号}
若在论文写作过程中有分条的需求可利用itemize或者enumerate 环境实现分条和交叉引用。itemize 可以满足自定义的序号标识,而enumerate则可以自动生成有序的标号。类似的\LaTeX 中也包含有定理定义证明等环境可以自动生成序号以及可以实现交叉引用,但是由于在期刊论文中的实用性不高,因此本文不予详细介绍。

\section{多级标题以及文件导入} \label{多级标题}
\LaTeX 在编辑过程中,通过命令可以自动生成有序的标题,并可以在导言区对其格式进行设置。然而在写文章的过程中,没有人可以一气呵成,然而由于在利用\LaTeX 写作过程中,会出现很多命令,并且当篇幅过长的时候影响审阅,以及会使运行时间变得越来越长,因此将不同章节份文件保存,最后进行整合可以使文章看起来简介有序,并且在无需更改的时候可以先行注释不予运行,加快运行速度,提高编辑效率。 
\subsection{多级标题格式设置}
\LaTeX 默认有三级标题,而在实际论文的写作过程中很少用到四级标题,因为会使标题序号变得异常的繁琐且不美观,如果需要四级标题,可直接利用\ref{自动编号}%讲自动编号的章节
中的itemize 环境实现。

而三级标题在未使用ctex中文宏包之前,默认为左对齐。在使用ctex宏包之后依然为左对齐,但是在使用ctexart宏包之后会默认为居中对齐。由于本文使用的文件类型为ctexart因此在需要使一级标题左对齐的时候需要在导言区进行设置。各级标题及其对应名命令如下:\\
一级标题 section\\
二级标题 subsection\\
三级标题 subsubsection

对各级标题格式进行设置可以利用ctexsetup语句,但是此语句并非实现此功能的唯一方法,其他方法不予详细介绍。本文主要针对工业工程期刊的要求,对各级标题进行字体字号设置,此外针对标题与段落间距的设置也会进行简要介绍,由于笔者暂未发现其他设置需求,因此本文不予介绍。
\subsection{文件导入}
实现章节以单独文件的形式保存,仅需将章节及文本内容单独一tex类型文件保存,且在正文中利用input 命令或者include 命令进行载入。 需要注意的是,当章节内容单独保存的时候不需要(不能)添加导言区内容;此外,利用input 命令载入的时候,文本会直接在段落后载入,而include 载入的时候则会在新的页面生成段落章节。因此可以根据自己需要选择使用,但期刊论文一般使用input命令即可。